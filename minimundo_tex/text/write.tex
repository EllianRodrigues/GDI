\section{Descrição do mundo real}

Um hotel é um lugar para viajantes, turistas ou qualquer pessoa que precise um lugar para ficar durante um período determinado de tempo. Os hoteis possuem quartos e uma variedades de serviços adicionais para melhorar a estadia dos hóspedes, tais como:

\begin{itemize}
    \item Serviços de alimentação;
    \item Serviços de limpeza;
    \item Comodidades de lazer;
    \item Segurança.
\end{itemize}

Existem hotéis de diversos tamanhos e categorias, desde de pequenos hoteis familiares até grandes redes internacionais de hoteis. São essenciais na indústria do turismo e necessitam de um bom gerenciamento para terem um selo de qualidade e proporcionar uma experiência agradável para os hóspedes.

Gerenciar um hotel ou resort necessita seguir vários padrões de qualidade que garantem o bom fucionamento das operações diários e a satisfação dos hóspedes, países como Espanha tem uma grande lista de requisitos para certificações de qualidade de gerenciamento de hoteis e resorts e varios outros serviços de turismo, essas certificações são emitidas pelo Spanish Institute for Tourismo Quality(ICTE), esses requisitos foram utilizados como material de base para esse documento.

Então eles foram dividos em diferentes áreas, incluindo gestão de quartos, reservas, hospedes, pagamentos, funcionários, instalações, serviços adicionais e administração geral. Sendo esses os principais pontos que os hoteis de qualidade focam o nosso objetivo é automatizar, facilitar e garantir a qualidade desses objetivos.

\newpage

\section{Objetivos da Aplicação}

\subsection{Descrição da aplicação}

A aplicação de gerenciamento de hotéis e resorts tem como objetivo criar uma plataforma auxiliar a administração do hotel a gerenciar os aspectos operacionais de um hotel ou resort. Ela modela às necessidades específicas de diferentes tipos de usuários: funcionários e hospedes. Essas funcionalidades são:

\subsection{Descrição das atividades}

\begin{enumerate}
    \item Gestão de Quartos
    \begin{itemize}
        \item Permitir que os funciónarios visualizem informações detalhas sobre todos os quartos disponíveis, incluindo tipo de quarto(Visto que normalmente hotéis possuem N tipos de quartos com características parecidas).
        \item Mantem um controle da quantidade de quartos disponíveis e reservados em tempo real.
    \end{itemize}
    \item Gestão de reservas
    \begin{itemize}
        \item Facilitar a criação, modificação e cancelamento de reservas por partes dos funcionários.
        \item Permitir que os hospedes reservam por períodos determinados e visualizem os detalhes de suas reservas.
        \item Gerenciar informações detalhadas sobre cada reserva, como datas de inícios e término, status de reserva e número de hóspede.
    \end{itemize}
    \item Gestão de Hospedes e Hóspedes
    \begin{itemize}
        \item Armazenas e gerenciar informações detalhadas dos hospedes, incluindo dados pessoais, detalhes de pagamento e histórico de reservas.
        \item Armazenar informações de hóspedes que não são hospedes pagantes, os acompanhantes, permitindo um gerenciamento adequado de todos os visitantes do resort.
    \end{itemize}
    \item Gestão de pagamentos
    \begin{itemize}
        \item Capturar e armazenas informações detalhadas sobre os pagamentos realizados pelos hospedes, incluindo métodos de pagamento, valor, data e detalhes de transação.
    \end{itemize}
    \item Gestão de funcionários
    \begin{itemize}
        \item Armazenar informações sobre funcionários, como cargo, salário e outros dados relevantes.
        \item Permitir que os funcionários com cargo administrador gerenciem a equipe de forma eficaz, visualizando detalhes dos funcionários e monitorando suas atividades.
    \end{itemize}
    \item Gestão Administrativa
    \begin{itemize}
        \item Fornecer as admnistradores controle total sobre os conteúdos do sistema, permitindo adcionar, editar e deletar informações sobre quartos, hospedes, reservas e pagamentos.
    \end{itemize}
    \item Serviços adicionais para Hóspedes
    \begin{itemize}
        \item Permitir que hóspedes reservem serviços adcionais oferecidos pelo resort, como spa, restaurante e passeios.
        \item Armazenas informações sobre esses serviços, incluindo descrição, preços e reservas associadas
    \end{itemize}
\end{enumerate}

\subsection{Benefícios Esperados}
\begin{itemize}
    \item \textbf{Eficiência Operacional:} Centralizar a gestão de quartos, reservas, pagamentos e funcionários em uma única plataforma, melhorando a eficiência e reduzindo erros.
    \item \textbf{Melhor experiência para os Hospedes:} Oferecer aos hospedes uma interface intuitiva para gerenciar suas reservas e informações pessoais.
    \item \textbf{Gerenciamento Simplificado:} Proporcionar aos administradores ferramentas para gerenciar todas as operações do resort, garantindo uma administração mais eficiente e eficaz.
    \item \textbf{Serviços personalizados:} Permitir numa plataforma centralizada a contratação de serviços adcionais para os hóspedes aumentando a atratividade e receita dos hósteis.
\end{itemize}

Ao atender esses objetivos, a aplicação de gerenciamento de hotéis e resorts será uma ferramenta vital para melhorar a operação diária e a experiência geral dos hóspedes em um grande resort.

A aplicação de gerenciamento de hotéis e resorts foi desenvolvida com base em nos seguintes artigos \cite{design_and_implemntation_of_hotel_room_management_system}. \cite{hotel_management_system_GUI}, \cite{design_of_hotel_management_system} na implementação de sistemas de hóteis e \cite{quality_management_in_hotels} sobre qualidade em gerenciamento de hosteis, que proporcionou insights valiosos para o modelo de sistema utilizado. Esses estudos ajudaram a moldar a arquitetura e as funcionalidades específicas necessárias para atender às complexas demandas operacionais e de serviço de um grande resort, visando melhorar tanto a eficiência operacional quanto a experiência geral dos hóspedes.

\newpage

\section{Descrição das entidades}

\subsection{Usuário}

A entidade usuário é armazena as informações básicas de todos os usuários do sistema.

\begin{itemize}
    \item ID (atributo identificador)
    \item Nome
    \item Telefone (atributo multivalorado)
    \item Email
    \item Endereço (Atributo Composto que se subdivide em Rua, Número,
Complemento, Bairro, Cidade, Estado, País, CEP)
    \item Data de nascimento
    \item Genero
\end{itemize}

\subsection{Hospede}

Entidade que herda de pessoa e representa os hospedes pagantes do hotel.

\begin{itemize}
    \item CPF (Atributo Identificador)
    \item Profissão
    \item Reservas
    \item Gênero
    \item Data de nascimento
\end{itemize}

\subsection{Visita}

Entidade fraca que depende do hospede.

\begin{itemize}
    \item Nome
    \item CPF*
    \item Idade
    \item Gênero
    \item Data da visita
    \item Data de nascimento
\end{itemize}

\subsection{Funcionário}

Entidade que herda de pessoa e representa os funcionários do hotel

\begin{itemize}
    \item CPF
    \item Salário.
    \item Data de contratação.
    \item Cargo
\end{itemize}

\subsection{Tipo de quarto}

Armazena informações sobre os diferentes tipos de quartos disponíveis no hotel.

\begin{itemize}
    \item CategoriaID (Atributo identificador)
    \item Nome
    \item Area (Área do tamanho do quarto)
    \item Numero de camas
    \item Numeros de ar-condicionados
    \item Numero de TVs
    \item Quantidade de quartos
    \item Numero de banheiros
    \item Preço
\end{itemize}

\subsection{Quarto}

Armazena informações específicas de cada quarto no hotel.

\begin{itemize}
    \item Numero do quarto (Atributo identificador)
    \item Nome\_Tipo (Refêrencia a "CategoriaID" na tabela 'Tipo de quarto')
    \item Reservado
\end{itemize}

\subsection{Reserva}

Armazena informações sobre as reservas realizadas pelos hospedes.

\begin{itemize}
    \item ReservaID (Atributo identificador)
    \item CPF (Referente ao atributo identificador da tabela "Hospedes")
    \item Numero do quarto (Referente ao "Numero do quarto" da tabela "Quartos")
    \item Check in
    \item Check out
    \item Status (Ativa, reservada, finalizada e cancelada)
    \item Informações de pagamentos (Atributo composto).
    \item Numero de hóspedes
\end{itemize}

\subsection{Serviço de Hóspede}

Armazena as informações sobre serviços adicionais oferecidos pelo hotel

\begin{itemize}
    \item Nome
    \item Descrição
    \item Preço
    \item ReservationID (Refêrencia a "ReservaID" na tabela "Reserva")
\end{itemize}

\newpage

\section{Descrição dos relacionamentos}

\subsection{Contratar}

Um hospedes contrata uma reserva de um quarto (Relacionamento triplo).

\textbf{Entidades envolvidas: }Hospedes, Reserva, Quarto.

\textbf{Relacionamento: } N:N:N.

\subsection{Reservar serviço de hóspedes}

Relacionamento entre a entidade associativa "Contratar" e serviços de hóspede.

\textbf{Entidades envolvidas: }Reserva, Serviço de Hóspede.

\textbf{Relacionamento: } N:N.

\subsection{Relação Tipo de Quarto e Quarto}

Um tipo de quarto pode ter vários quartos associados a ele e um quarto só pode pertencer a um único tipo de quarto.

\textbf{Entidades envolvidas: }Tipo de quarto e Quarto.

\textbf{Relacionamento: } 1:N.

\subsection{Supervisão}

Um funcionário pode supervisionar outros funcionários (Auto-relacionamento).

\textbf{Entidade envolvida: }Funcionário.

\textbf{Relacionamento: } 1:N.

% \subsection{Limpeza}

% Um Funcionário limpa um Quarto em um determinado período, quando o quarto não está reservado. (Relacionamento temporal)

\textbf{Entidades envolvidas: }Funcionário e Quarto.

\textbf{Relacionamento: } N:N

\textbf{Atributos adicionais: }Data de limpeza, Hora de início, hora de término.

\newpage

\section{Possíveis perguntas}
\subsection{Perguntas Gerais}
\begin{enumerate}
    \item \textbf{Perguntas de Contratos}
    \begin{itemize}
        \item Quais contratos com hóspedes estão ativos atualmente?
        \item Quando termina o contrato do hóspede Ana Lima?
    \end{itemize}
    \item \textbf{Sobre os Quartos}
    \begin{itemize}
        \item Quais quartos estão atualmente disponíveis?
        \item Quantos quartos do tipo "Deluxe" estão disponíveis?
        \item Qual é o status atual do quarto 101?
        \item Quais são as comodidades oferecidas nos quartos "Premium"?
    \end{itemize}
    \item \textbf{Sobre Reservas}
    \begin{itemize}
        \item Quais reservas estão previstas para check-in hoje?
        \item Há alguma reserva para o próximo fim de semana?
        \item Qual é o status da reserva do hóspede João Silva?
    \end{itemize}
\end{enumerate}

\subsection{Perguntas dos hospedes}
\begin{enumerate}
    \item \textbf{Sobre Reservas}
    \begin{itemize}
        \item Quais são minhas reservas futuras?
        \item Qual é o status da minha reserva para o próximo mês?
        \item Posso ver o histórico das minhas reservas?
    \end{itemize}
    \item \textbf{Sobre Pagamentos}
    \begin{itemize}
        \item Qual é o valor total que já paguei ao hotel?
        \item Quanto eu gastei no hotel na reserva específica da semana passada?
        \item Quanto eu gastei no hotel ao total no ano passado?
    \end{itemize}
    \item \textbf{Sobre Serviços Adicionais}
    \begin{itemize}
        \item Há algum serviço adicional disponível durante a minha estadia?
        \item Quais serviços adicionais eu já utilizei?
        \item Quanto custou o serviço de spa que reservei?
    \end{itemize}
\end{enumerate}

\subsection{Perguntas dos funcionários}
\begin{enumerate}
    \item \textbf{Sobre hóspedes}
    \begin{itemize}
        \item Quantos hóspedes estão presentes no hotel agora?
    \end{itemize}
\end{enumerate}

\newpage

\section{Possíveis relatórios}

\subsection{Relatórios Gerais} 
\begin{enumerate}
    \item \textbf{Relatório de Contratos}
    \begin{itemize}
        \item Detalhes dos contratos ativos e finalizados com hóspedes.
        \item Detalhes de cada contrato (Quem contratou e qual quarto especificio).
    \end{itemize}
    \item \textbf{Relatório de Ocupação}
    \begin{itemize}
        \item Taxa de ocupação por período (diário, semanal, mensal).
        \item Comparação de ocupação entre diferentes tipos de quartos.
        \item Tendências de ocupação em certos períodos de tempo.
    \end{itemize}
    \item \textbf{Relatório de Funcionários}
    \begin{itemize}
        \item Detalhes de cada funcionário (cargo, salário, data de contratação).
        \item Quantos funcionários ele supervisiona.
    \end{itemize}
    \item \textbf{Relatório de Serviços de Hóspedes}
    \begin{itemize}
        \item Serviços adicionais mais populares.
        \item Receita gerada por serviços adicionais.
        \item Informações sobre os hóspede que contratam aquele serviço (Genero e idade).
    \end{itemize}
    \item \textbf{Marketing}
    \begin{itemize}
        \item Dados detalhados dos hóspedes.
        \item Histórico de reservas.
        \item Histórico de serviços adcionais que aquele hóspede reserva.
        \item Somátorio dos pagamentos feitos pelo hóspede.
    \end{itemize}
\end{enumerate}

\subsection{Relatórios para Hóspedes}
\begin{enumerate}
    \item \textbf{Relatório de Reservas}
    \begin{itemize}
        \item Visualizar seu histórico de reservas.
        \item Detalhes de reservas atuais e futuras.
        \item Status das reservas (ativa, cancelada, finalizada).
    \end{itemize}
    \item \textbf{Relatório de Pagamentos}
    \begin{itemize}
        \item Histórico de pagamentos realizados.
        \item Detalhes das transações (data, valor, método de pagamento).
        \item Informações de pagamento.
    \end{itemize}
    \item \textbf{Relatório de Serviços Adicionais}
    \begin{itemize}
        \item Serviços adicionais feitos na sua reserva.
        \item Detalhes dos serviços (Nome, descrição e preço).
    \end{itemize}
\end{enumerate}

\subsection{Relatórios para funcionários}
\begin{enumerate}
    \item \textbf{Relatório de Disponibilidade de Quartos}
    \begin{itemize}
        \item Quartos disponíveis por tipo.
        \item Quartos ocupados e seu status de reserva.
    \end{itemize}
    \item \textbf{Relatório de Reservas}
    \begin{itemize}
        \item Reservas ativas, canceladas e finalizadas.
        \item Detalhes de reservas futuras e check-ins previstos.
        \item Reservas por período.
    \end{itemize}
    \item \textbf{Relatório de Hóspedes}
    \begin{itemize}
        \item Serviços contratados por hóspedes.
        \item Número de hóspedes por reserva.
    \end{itemize}
\end{enumerate}

% Exemplo de código

% Incrível "Hello World" em python em O(n!)

% \lstinputlisting[caption={Exemplo de código em Python.}, label={lst:python_code}]{codes/hello_world.py}
